\documentclass{llncs}

\usepackage[font=footnotesize]{caption} 
\usepackage{url}
\usepackage{graphicx} 
\usepackage{caption} 
\usepackage{hyperref}
\usepackage[T1]{fontenc} 
\usepackage[utf8]{inputenc}
\usepackage[multiple]{footmisc} 
\usepackage{algorithm2e}
\usepackage{mathtools}

\pagestyle{plain}

\begin{document}

\mainmatter % start of the contributions

\title{Concept Drift: Status Report}

\titlerunning{Concept Drift: Status Report}

\author{Albert Mero\~{n}o-Pe\~{n}uela\inst{1,2}}

\authorrunning{Albert Mero\~{n}o-Pe\~{n}uela}
% abbreviated author list (for running head)
%
%%%% list of authors for the TOC (use if author list has to be modified)
\tocauthor{Albert Mero\~{n}o-Pe\~{n}uela et al.}
%
% \institute{Department of Computer Science, VU University
% Amsterdam, The
% Netherlands\\ \email{\{author\}@vu.nl}}

\institute{Department of Computer Science, VU University Amsterdam, NL
  \email{albert.merono@vu.nl} 
  \and Data Archiving and Networked Services, KNAW, NL}

\maketitle

\begin{abstract}
In the Semantic Web, \emph{concepts} are the central constructs that
are used to describe sets of objects with shared
characteristics. Although it is widely assumed to be stable, the
meaning of these concepts can change over time: we call this
\emph{concept drift}. This document reports the status of a research
plan to study concept drift in the Semantic Web.
\keywords{Semantic Web, Concept Drift}
\end{abstract}

\section{Introduction}
\label{sec:intro}

Problem Statement: What problem are you trying to solve?

Relevancy: Why is the problem relevant?

The remaining of this paper is organised as follows.

\section{Related Work}
\label{sec:related-work}

\section{Research Questions}

Research Questions: What are the research questions that you plan to
address?

The main research question this plan addresses is:

\begin{quotation}
What are the causes and consequences of the variation of meaning in
concepts on the Semantic Web?
\end{quotation}

We structure this question in multiple specific research questions:

\begin{itemize}
  \item How stable is the meaning of concepts in the Semantic Web at
    its full scale?
\end{itemize}

\section{Hypotheses}
Hypotheses: What hypotheses are related to you research questions?

\section{Approach}
Approach: How are you planning to address your research questions and
test your hypotheses?

\section{Reflections}
Reflections: Why do you think you will succeed where others failed?

\section{Evaluation Plan}
Evaluation plan: How will you measure your success - faster/more
accurate/less failures/etc.?

\section{Conclusions}
\label{sec:conclusions}


\bibliographystyle{splncs03}
%\bibliographystyle{abbrv}
\bibliography{cd}
\end{document}

